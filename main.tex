\documentclass[11pt]{article}
\usepackage{graphicx} % Required for inserting images
\setlength{\parindent}{0pt}
\usepackage{hyperref}
\usepackage{enumitem}
\usepackage[utf8]{inputenc}
\usepackage[T1]{fontenc}
\usepackage[brazil]{babel}
\usepackage{lipsum}
\usepackage[left=1.06cm,top=1.7cm,right=1.06cm,bottom=0.49cm]{geometry}

\begin{document}
\begin{center}
    \textbf{Jorge Enciso}\\
    \begin{center}
        Machine Learning Engineer and Performance Engineer
    \end{center}
    \hrulefill
\end{center}

\begin{center}
    Asunción, Paraguay \textbullet \ jorged.encyso@gmail.com \textbullet \ +595 981 631462
\end{center}

\vspace{0.5pt}

\begin{center}
    \textbf{Education}
\end{center}
\textbf{Colegio Japonés Paraguayo} \hfill Asunción, Paraguay

High School Diploma, Natural Sciences Track, GPA [10/10], Valedictorian \hfill Graduation Date Thesis [Nov 2023]

\begin{itemize}%
    \item Relevant Coursework: Advanced Mathematics, Physics, Mechatronics
    \item Honors Thesis: \textit{“Niche Modeling with Deep Learning”}
        \begin{itemize}
            \item Created an automated training pipeline to reconstruct species’ spatial distributions from historical occurrence data.
            \item Developed and benchmarked models for generalized distribution estimation (Neural Networks, Logistic Regression, Naive Bayes, SVM, Random Forest) and pseudo-absence generation (One-Class SVM, K-means Clustering).
        \end{itemize}
\end{itemize}

\vspace{12pt}

\begin{center}
    \textbf{Research Experience}
\end{center}

\textbf{Multi-modal modelling for Geophysical Forecasting} \hfill Polars, Pandas, Pytorch \\
\textit{MHD-informed Multi-Modal Networks for Geomagnetic Forecasting} \hfill October 2023 – May 2024
\begin{itemize}[noitemsep]
    \item Developed a multi-modal ML framework using data from LASCO, SDO, ACE, and DSCOVR to predict Dst index and magnetospheric boundaries.
    \item Engineered transformer-based architecture for satellite time series fusion; integrated MHD constraints into training dynamics.
    \item Led all aspects: data preprocessing, model design, CUDA kernel optimization, and evaluation.
\end{itemize}

\textbf{Computer Vision for Astronomic instrumentation} \hfill OpenCV, Scikit-Image, Pytorch \\
\textit{CorKit: A Framework for LASCO Coronagraph Calibration} \hfill March 2024 – July 2024
\begin{itemize}[noitemsep]
    \item Developed a pipeline for automated image preprocessing and calibration of solar coronagraph data.
    \item Integrated physics-informed constraints to correct lens artifacts and vignetting in LASCO C2/C3 data.
    \item Implemented using PyTorch and OpenCV; optimized modules for GPU acceleration.
\end{itemize}

\textbf{Statistical Mechanics informed Neural Networks} \hfill NVIDIA Modulus, Pytorch, Polars, Pandas \\
\textit{Vlasov-Maxwell informed Operator Learning for Solar Wind Modeling} \hfill October 2024 – Present
\begin{itemize}[noitemsep]
    \item Modeled solar wind behavior using physics-informed machine learning.
    \item Incorporated statistical mechanics constraints into the governing equations.
    \item Designed custom Physics-Informed Neural Operators (PINOs) tailored to the system.\end{itemize}

\textbf{Machine Learning for Celestial Mechanics} \hfill NVIDIA Modulus, Pytorch, Numpy \\
\textit{PINNs for Celestial Mechanics: The Family of N-body problems} \hfill December 2024 – Present
\begin{itemize}[noitemsep]
    \item Understanding chaotic systems with Physics informed Neural Networks.
    \item Enforced hamiltonian constraints through physics-informed training loss functions.
    \item Studied resonancy and periodic properties of the physical system.
\end{itemize}

\begin{center}
    \textbf{Specialized Projects}
\end{center}

\textbf{LightTorch – Deep Learning Framework} \hfill Lightning, Pytorch, CUDA C++ \\
\textit{Personal ML Framework Project} \hfill 2024 – Present
\begin{itemize}[noitemsep]
    \item Wrapper over the Pytorch Lightning and plain Pytorch backends.
    \item Abstractions for supervised, self-supervised, and adversarial training.
    \item More than 20 modules extending novel architectures, including Partial Convolutions, Fourier Layers, and Attention mechanisms, among others.
    \item In progress: Low level optimizations for non-standard modules.
\end{itemize}

\textbf{Fusion – IO-Aware Kernelized Training Compiler} \hfill CUDA, C++, Pytorch\\
\textit{In Development} \hfill 2024
\begin{itemize}[noitemsep]
    \item Design a programming language syntactically parallel as a machine learning standard.
    \item Create a new version of the forward-backward standard that exploits IO operation latency.
    \item Features include stream-aware optimization, prefetching graphs for latency hiding, and compatability with state-of-the-art frameworks.
\end{itemize}

\textbf{Fourier is All You Need} \hfill CUDA, C++, Pytorch \\
\textit{Machine Learning Architecture} \hfill 2024
\begin{itemize}[noitemsep]
    \item Exploited mathematical properties of the Fourier Transform to accelerate Convolutional Neural Networks.
    \item Investigated the properties of the Fourier space in training scenarios.
    \item Designed UNet-Like variations for the Fourier Layer case.
\end{itemize}

\begin{center}
    \textbf{Miscellaneous Projects}
\end{center}

\textbf{Rapid Eye Movement Detection} \hfill Pytorch, Pandas, Numpy, Scikit-Learn \\
\textit{Sleeping stages classification} \hfill 2024
\begin{itemize}[noitemsep]
    \item Developed models to classify sleeping stages.
    \item Achieved state-of-the-art classification with 97\% F1 Score.
    \item Implemented Deep Recurrent Neural Networks and Attention-based architectures.
\end{itemize}

\textbf{SADI A.I.} \hfill Pytorch, OpenCV \\
\textit{Security AI} \hfill 2024
\begin{itemize}[noitemsep]
    \item Developed foundational models to classify and recognize threats.
    \item Trained and fined-tuned state-of-the-art Computer Vision models (YOLO, YOLO-NAS).
    \item Real-Time face detection, threat detection, face recognition, and individual detection.
\end{itemize}

\textbf{Sketcher A.I.} \hfill Pytorch, OpenCV \\
\textit{Art AI} \hfill 2024
\begin{itemize}[noitemsep]
    \item Created foundational pipeline to create sketches from images.
    \item Developed state-of-the-art style-transfer mechanisms.
\end{itemize}

\begin{center}
    \textbf{Awards \& Recognitions}
\end{center}

\begin{itemize}[noitemsep]
    \item Silver Medal – National Math Olympiad (8th/9th Division), Paraguay, 2020
    \item Bronze Medal – National Math Olympiad (10th/11th/12th Division), Paraguay, 2021
    \item Gold Medal – Regional Physics Olympiad (Advanced Level), Paraguay, 2022
    \item Silver Medal – National Physics Olympiad (Intermediate Level), Paraguay, 2022
    \item National Delegate – Ibero-American Physics Olympiad, 2022
    \item Honorable Mention – NASA Space Apps Challenge Paraguay, 2023
    \item 2nd Place in Technology – Marie Curie National Science Fair, 2023
    \item 2nd Place - 1st National Stratospheric Platforms Contest, 2023
\end{itemize}


\begin{center}
    \textbf{Leadership \& Experience}
\end{center}

\textbf{DDS.py}	\hfill Asuncion, Paraguay

\textbf{Founder, Teacher} \hfill July 2023 – December 2024
\begin{itemize}[noitemsep, topsep=0pt, partopsep=0pt, parsep=0pt]
  \item Taught foundational programming (Python) and ML concepts to high school students.
  \item Developed and maintained a public learning roadmap on GitHub.
\end{itemize}

\textbf{Mechatronics Teacher}	\hfill Asuncion, Paraguay

\textbf{Role} \hfill March 2024 – November 2024
\begin{itemize}[noitemsep, topsep=0pt, partopsep=0pt, parsep=0pt]
  \item General purpose progamming (C++, Python) lectures for High School students.
  \item Taught Networking (OSI Model, TCP/UDP communication), Astronomy software design patterns (NASA Software Engineering book), serial protocols (USB, $I^{2}C$, etc).
  \item Taught programming paradigms: Object Oriented Programming (OOP) and Functional programming.
\end{itemize}

\textbf{Physics Teacher}	\hfill Asuncion, Paraguay

\textbf{Role} \hfill March 2023 – November 2024
\begin{itemize}[noitemsep, topsep=0pt, partopsep=0pt, parsep=0pt]
  \item Taught lectures on classical mechanics and electromagnetism for physics olympiads.
\end{itemize}


\begin{center}
    \textbf{Certifications}
\end{center}

\textbf{Stanford University: Machine Learning}: \url{https://www.coursera.org/account/accomplishments/specialization/certificate/VNCPL4MXPB5A}

\textbf{IBM AI Engineernig}: \url{https://www.coursera.org/account/accomplishments/specialization/certificate/JEYSUHXTCYU5}

\textbf{Jovian: Zero to GANs with Pytorch}: \url{https://jovian.com/certificate/MFQTQMZXGM}

\textbf{Duoling English Test}: \url{https://certs.duolingo.com/vc54x03gh7zcprnd}

\begin{center}
    \textbf{Skills}
\end{center}

\textbf{Languages:} Spanish (Native), English (Fluent), Portuguese (Basic), Japanese (Basic)

\textbf{Programming:} C++, CUDA C++, Python, Rust, Bash, Lua

\textbf{Scientific:} Deep Learning, Numerical Optimization, Compiler Design, Geospatial ML, Remote Sensing, Data Fusion

\textbf{ML Tools:} CuDNN, CUDA Toolkit, Triton, PyTorch, Lightning, NVIDIA Modulus, Scikit-learn

\textbf{Data Tools:} Pandas, Polars, Matplotlib, Seaborn, PySpark, cudf, SQL

\end{document}
