\documentclass[11pt]{article}
\usepackage{graphicx} % Required for inserting images
\setlength{\parindent}{0pt}
\usepackage{enumitem}
\usepackage[utf8]{inputenc}
\usepackage[T1]{fontenc}
\usepackage[english]{babel}
\usepackage{lipsum}
\usepackage[left=1.06cm,top=1.7cm,right=1.06cm,bottom=0.49cm]{geometry}
\usepackage{bibentry}
\usepackage{hyperref}

\begin{document}
\begin{center}
    \textbf{Jorge Enciso}\\
    \begin{center}
        Machine Learning Engineer and Machine Learning Performance Engineer
    \end{center}
    \hrulefill
\end{center}

\begin{center}
    Asunción, Paraguay • \href{mailto:jorged.encyso@gmail.com}{jorged.encyso@gmail.com} • +595 981 631462 • \href{https://github.com/jorgedavyd}{Github} • \href{https://linkedin.com/in/jorge-david-enciso-martínez-149977265}{Linkedin} • \href{https://medium.com/@jorged.encyso}{Medium}
\end{center}

\vspace{0.5pt}

\begin{center}
    \textbf{Education}
\end{center}
\textbf{Colegio Japonés Paraguayo} \hfill Asunción, Paraguay

High School Diploma, Natural Sciences Track, GPA [10/10], Valedictorian \hfill Graduation Date Thesis [Nov 2023]

\begin{itemize}%
    \item Relevant Coursework: Advanced Mathematics, Physics, Mechatronics
    \item Honors Thesis: \textit{“Niche Modeling with Deep Learning”}
        \begin{itemize}
            \item Created an automated training pipeline to reconstruct species’ spatial distributions from historical occurrence data.
            \item Developed and benchmarked Machine Learning and Deep Learning models for generalized distribution estimation (Neural Networks, Logistic Regression, Naive Bayes, SVM, Random Forest)
            \item Three step pseudo-absence generation with unsupervised methods (One-Class SVM, K-means Clustering) \cite{pseudo} implemented from scratch, random and radius based sampling.
        \end{itemize}
\end{itemize}

\vspace{12pt}

\begin{center}
    \textbf{Research Experience}
\end{center}

\textbf{Multi-modal modelling for Geophysical Forecasting} \hfill Polars, Pandas, Pytorch \\
\textit{MHD-informed Multi-Modal Networks for Geomagnetic Forecasting} \hfill October 2023 – May 2024
\begin{itemize}[noitemsep]
    \item Developed Deep Learning architectures for long-term Solar Wind forecasting.
    \item Engineered Bahdanau-based architectures for satellite time series fusion.
    \item Integrated MHD constraints as a physics informed cost function.
    \item Created the concept of "Encoder Forcing", a technique to enforce the calibration of the input data throughout the training process.
    \item Achieved 96\% accuracy, 86\% recall, and 93\% precision, competing with state-of-the-art.
\end{itemize}

\textbf{Computer Vision for Astronomic instrumentation} \hfill OpenCV, Scikit-Image, Pytorch \\
\textit{CorKit: A Framework for LASCO Coronagraph Calibration} \hfill March 2024 – July 2024
\begin{itemize}[noitemsep]
    \item Developed an alternative pipeline for automated image preprocessing and calibration of solar coronagraph data.
    \item Implemented the original pipeline \cite{lasco} from scratch with modern programming languages with extended linear algebra support (Python, C++).
    \item Created a Multi-Layered UNet-Like Partial Convolutional Neural Network architecture from the foundational Partial CNN work \cite{conv} from scratch.
    \item Increased the reliability of coronagraph scientific products with contextual image inpainting.
\end{itemize}

\textbf{Statistical Mechanics informed Neural Networks} \hfill NVIDIA Modulus, Pytorch, Polars, Pandas \\
\textit{Vlasov-Maxwell informed Operator Learning for Solar Wind Modeling} \hfill October 2024 – Ongoing
\begin{itemize}[noitemsep]
    \item Modeled solar wind's electron's anisotropic behavior using physics-informed machine learning.
    \item Implemented Deep Neural Networks, Recurrent Neural Networks, and Neural Operators to replicate electron's probability density function.
    \item Unveiled reliable data-driven distributions for an obscure area in Plasma Physics.
\end{itemize}

\textbf{Machine Learning for Celestial Mechanics} \hfill NVIDIA Modulus, Pytorch, Numpy \\
\textit{PINNs for Celestial Mechanics: The Family of N-body problems} \hfill December 2024 – Ongoing
\begin{itemize}[noitemsep]
    \item Modeled gravitational interactions between celestial bodies with Physics informed Neural Networks \cite{pinn}.
    \item Enforced hamiltonian constraints of the N-Free body problem and its variations (Euler, Lagrange) using Lagrangian optimization and symplectic integrators.
    \item Presented PINNs as an alternative to reliably study celestial mechanics without overly complex numerical methods.
\end{itemize}

\begin{center}
    \textbf{Specialized Projects}
\end{center}

\textbf{LighTorch – Deep Learning Framework} \hfill Lightning, Pytorch, Optuna, CUDA C++ \\
\textit{Personal ML Framework Project} \hfill 2024 – Ongoing
\begin{itemize}[noitemsep]
    \item Wrapper over the Pytorch Lightning and plain Pytorch backends.
    \item Abstractions for supervised, self-supervised, and adversarial training.
    \item Multi-Objective optimization for Hyperparameter tuning with Optuna.
    \item More than 20 modules extending novel architectures, including the Partial Convolutions \cite{conv}, Fourier Layers (including deconvolutions) \cite{fourier}, among others.
    \item Implemented additional architectures from scratch: Rotary Positional Encoding \cite{pe}; RMS Norm \cite{norm}; GLU Variants \cite{ffw}; Multi Query Attention, Grouped Query Attention \& Multi Head Attention \cite{attention}.
    \item In progress: Low level optimizations for non-standard modules.
\end{itemize}

\textbf{Fusion – IO-Aware Kernelized Training Compiler} \hfill CUDA, C++, Pytorch\\
\textit{In Development} \hfill 2024 - Ongoing
\begin{itemize}[noitemsep]
    \item Design a programming language syntactically parallel as a machine learning standard.
    \item Create a new version of the forward-backward standard that exploits IO operation latency.
    \item Features include stream-aware optimization, prefetching graphs for latency hiding, and compatibility with state-of-the-art frameworks.
\end{itemize}

\textbf{Fourier is All You Need \& Fourier Variational Autoencoders} \hfill CUDA, C++, Pytorch \\
\textit{Machine Learning Architecture} \hfill 2024
\begin{itemize}[noitemsep]
    \item Exploited mathematical properties of the Fourier Transform to accelerate Convolutional Neural Networks.
    \item Investigated the properties of the Fourier space in training scenarios.
    \item Designed UNet-Like variations for the Fourier Layer case \cite{fourier} (implemented from scratch).
    \item Created a Fourier Layer alternative for UNet-Like autoencoders: Fourier Deconvolutional Layer.
\end{itemize}

\begin{center}
    \textbf{Miscellaneous Projects}
\end{center}

\textbf{Rapid Eye Movement Detection} \hfill Pytorch, Pandas, Numpy, Scikit-Learn \\
\textit{Sleeping stages classification} \hfill 2023
\begin{itemize}[noitemsep]
    \item Implemented Recurrent Neural Networks to classify sleeping stages.
    \item Achieved state-of-the-art classification with 97\% F1 Score (Residual GRU model).
    \item Deep Recurrent Neural Networks and Attention-based architectures.
\end{itemize}

\textbf{SADI A.I.} \hfill Pytorch, OpenCV \\
\textit{Security AI} \hfill 2023
\begin{itemize}[noitemsep]
    \item Developed foundational models to classify and recognize threats.
    \item Trained state-of-the-art Computer Vision models (YOLO, YOLO-NAS).
    \item Real-Time face detection, threat detection, face recognition, and individual detection.
\end{itemize}

\textbf{Sketcher A.I.} \hfill Pytorch, OpenCV \\
\textit{Art AI} \hfill 2023
\begin{itemize}[noitemsep]
    \item Created foundational pipeline to create sketches from images.
    \item Implemented state-of-the-art style-transfer mechanisms.
\end{itemize}

\begin{center}
    \textbf{Awards \& Recognitions}
\end{center}

\begin{itemize}[noitemsep]
    \item Silver Medal – National Math Olympiad (8th/9th Division), Paraguay, 2020
    \item Bronze Medal – National Math Olympiad (10th/11th/12th Division), Paraguay, 2021
    \item Gold Medal – Regional Physics Olympiad (Advanced Level), Paraguay, 2022
    \item Silver Medal – National Physics Olympiad (Intermediate Level), Paraguay, 2022
    \item National Delegate – Ibero-American Physics Olympiad, 2022
    \item Honorable Mention – NASA Space Apps Challenge Paraguay, 2023
    \item 2nd Place in Technology – Marie Curie National Science Fair, 2023
    \item 2nd Place - 1st National Stratospheric Platforms Contest, 2023
\end{itemize}


\begin{center}
    \textbf{Experience}
\end{center}

\textbf{DDS.py}	\hfill Asuncion, Paraguay
\hfill July 2023 – December 2024
\begin{itemize}[noitemsep]
  \item Taught foundational programming (Python) and ML concepts to high school students.
  \item Developed and maintained an alternative learning roadmap on GitHub.
  \item An introductory to advance course on deep learning, and an introduction to ML performance engineering.
\end{itemize}


\textbf{Mechatronics Teacher}	\hfill Asuncion, Paraguay
\hfill March 2024 – November 2024
\begin{itemize}[noitemsep]
  \item General purpose programming (C++, Python) lectures for High School students.
  \item Taught Networking (OSI Model, TCP/UDP communication), Astronomy software design patterns (NASA Software Engineering book), serial protocols (USB, $I^{2}C$, etc).
  \item Taught programming paradigms: Object Oriented Programming (OOP) and Functional programming.
\end{itemize}

\textbf{Physics Teacher}	\hfill Asuncion, Paraguay
\hfill March 2023 – November 2024
\begin{itemize}[noitemsep]
  \item Taught lectures on classical mechanics and electromagnetism for physics olympiads.
\end{itemize}

\textbf{Oym Systems Group S.A.}	\hfill Asuncion, Paraguay

\textbf{ML \& Devops Engineer - Research Team} \hfill July 2024 – Ongoing
\begin{itemize}[noitemsep]
  \item Develop cloud infrastructure pipelines for ERP software scalability.
  \item Created LLM pipelines with Retrieval Augmented Generation (RAG) and Low Rank Adaptation (LoRA) for fine-tuning.
  \item Refactoring monolithic architectures into dockerized microservices.
  \item Implemented Kubernetes Operators for horizontal and vertical scalability.
  \item Proposed Continuous Integration and Deployment Git operations.
\end{itemize}

\textbf{Tigo - Telecel - Milicom}	\hfill Asuncion, Paraguay

\textbf{Machine Learning Engineer \& AI Specialist} \hfill June 2025 – Ongoing
\begin{itemize}[noitemsep]
  \item Finetuned 8B quantized LLMs on consumer-grade hardware for QLoRA PEFT finetuning for test case generation.
  \item Implemented brokers (for LangGraph) for Model Context Protocol (MCP).
  \item Implemented RAG retrievers (code, chat, text) as tooling for Agentic workflows.
  \item Develop cloud infrastructure pipelines for monitoring, security, and deployment of Agentic LLMs.
  \item Worked as a consultant for industry level AI applications.
\end{itemize}

\begin{center}
    \textbf{Certifications}
\end{center}

\begin{itemize}
    \item \href{https://www.coursera.org/account/accomplishments/specialization/certificate/VNCPL4MXPB5A}{\textbf{Stanford University: Machine Learning}}
    \item \href{https://www.coursera.org/account/accomplishments/specialization/certificate/JEYSUHXTCYU5}{\textbf{IBM AI Engineernig}}
    \item \href{https://jovian.com/certificate/MFQTQMZXGM}{\textbf{Jovian: Zero to GANs with Pytorch}}
    \item \href{https://certs.duolingo.com/vc54x03gh7zcprnd}{\textbf{Duoling English Test}}
\end{itemize}


\begin{center}
    \textbf{Skills}
\end{center}

\textbf{Languages:} Spanish (Native), English (Fluent), Japanese (Basic)

\textbf{Programming:} C++, CUDA C++, PTX ISA, Python, Rust, Bash, Lua, Nix

\textbf{Scientific:} Deep Learning, Numerical Optimization, Compiler Design, Geospatial ML, Remote Sensing, Data Fusion

\textbf{ML Tools:} CuDNN, CUDA Toolkit, Tensor RT, Triton, PyTorch, Lightning, NVIDIA Modulus, Scikit-learn, Transformers, LangChain, LangGraph, vLLM, NVIDIA NeMo

\textbf{Data Tools:} Pandas, Polars, Matplotlib, Seaborn, PySpark, cudf, SQL, Milvus

\textbf{Devops \& MLops: } Docker, Istio (Service mesh), Nginx, Helm, Kubernetes, Argo CD, NVIDIA NIM, Prometheus, Grafana, Tensorboard, Hydra

\bibliographystyle{plain}
\bibliography{references}

\end{document}
